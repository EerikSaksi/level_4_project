    
\documentclass[11pt]{article}
\usepackage{times}
    \usepackage{fullpage}
    
    \title{Collaborative Activity Tracker}
    \author{ {{Eerik Saksi}} - {{2392230s}} }

    \begin{document}
    \maketitle
    
    
     

\section{Status report}

\subsection{Proposal}\label{proposal}

\subsubsection{Motivation}\label{motivation}

\emph{{[}Inactivity is one of the modern leading killers, and despite all the science in favor of it, most do not exercise enough. This is largely due to the time commitment and discomfort caused by exercise. Many use activity trackers and apps to gamify exercise while others train with their friends. These make you accountable and make exercise funner. {]}}

\subsubsection{Aims}\label{aims}

\emph{{[} Who's to say that gamified progress and group training are mutually exclusive? My RPG app lets teams fight an enemy together. Damage is calculated based on exercise performance and effort. Gamified progress makes exercise more fun and satisfying, but it also has more purpose beyond a score: it helps your team progress to the next enemy. There is also social obligation and accountability to stay consistent and do your part, as well as outgroup competition. There is also a certain time in which you need to defeat the enemy in, which creates time pressure.{]}}

\subsection{Progress}\label{progress}
\begin{itemize}
\tightlist
\item Postgres database with tables containing data on users, groups, their performance, etc.
\item Set up Google authentication. Along side this, security policies on database rows based on identity 
\item Autogenerated GraphQL API from Postgres database using PostGraphile. Manually omitted some queries and fields, as well as created database functions that trigger when some queries are called. 
\item React Native (Expo) Typescript app that queries queries the PostGraphile API
\item Login through server
\item 8 different animated pixel enemies, that rotate over levels with ever increasing health
\item 6 different pixel animated avatars based on your relative strength (animated idle )
\item
\item Profile updates (gender, weight), exercise updates (new personal best), as well as workout updates. These 

\end{itemize}

\emph{{[}
    Briefly state your progress so far, as a bulleted list
    {]}}
\subsection{Problems and risks}\label{problems-and-risks}

\subsubsection{Problems}\label{problems}

\emph{{[}What problems have you had so far, that have held up the
project?{]}}

\subsubsection{Risks}\label{risks}

\emph{{[}What problems do you foresee in the future and how will you
mitigate them?{]}}

\subsection{Plan}\label{plan}

\emph{{[}Time plan, in roughly weekly to monthly blocks, up until
submission week{]}}

    
\subsection{Ethics and data}\label{ethics}
\emph{Specify what ethical approval you need to do your evaluation and how you are approaching it. This is mandatory. 
Specify what data you expect to collect in your evaluation. Explain how this data will help you evaluate your project.
}

Options for ethics:
\item This project does not involve human subjects or data. No approval required.
\item I have verified that the ethics checklist will apply to any evaluation I need to do. I will sign and complete the checklist.
\item I have sought ethical guidance from the School's ethics convener and I will:
\begin{itemize}
    \item Proceed under specific instructions from the Ethics convener (e.g. modified checklist).
    \item Apply for College Ethics Board approval.
    \item Other procedure (give details)
\end{itemize}    


\end{document}
